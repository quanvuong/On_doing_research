\documentclass[11pt]{article}

\usepackage[margin=0.3in]{geometry}
\usepackage{amsmath}
\usepackage{physics}
\usepackage{graphicx}

%Loading this so that mathbb is defined
\usepackage{amsfonts}

% This is required for triangleq
\usepackage{amssymb}

\begin{document}

\section{Why?}

Keeping a research notebook is the most important thing one can do to ensure steady progress in a research project. It is useful for two reasons:

\begin{enumerate}
	\item Ensure clarity of thought: it is impossible to write coherently without first thinking coherently. Keeping a research notebook is thus a way to self-verify whether one's thoughts about a subject are coherent.
	\item Keep track of intermediate results: in a research project, we develop many small intermediate results that eventually lead to the big insight. It is important to store these intermediate results in an easy-to-access way.
\end{enumerate}

\section{How?}

The qualities of a good research notebook are:

\begin{enumerate}
	\item Self-contained: except for established concepts and notations, the notebook should contain all background information needed to understand its content.
	\item Concise: A rambling research notebook indicates a confused state of mind.
	\item Neatly organized: as one of the purpose of the notebook is to store intermediate results, they should be written in such a way that allows for easy retrieval and double-checking of these results.
	\item Detailed: Each entry in the notebook should contain all details relevant to that entry, or references to such details. 
\end{enumerate}

I highlight the importance of the easy double-checking of the results. When we look back at the results, sometimes we forget how we obtained them and wanted to verify they are correct. If we do not store the results in a way that allows for easy double-checking, we will waste a lot of time in this step.
\\

For theoretical result, this involves writing the proof in a neat manner. For experimental result, this involves making it easy to reproduce the results of the experiments. I usually copy the code corresponding to one intermediate result into its own folder with meaningful folder name. 

\section{Template}

Each entry in the notebook can follow this template:

\begin{enumerate}
	\item Date
	\item What is it I will spend time on next? Example: reading a paper, running an experiment, trying to prove a result, etc.
	\item Why is this important? How is it relevant to the research goal?
	\item After I spent time on this, what did I learn? What are the expected and surprising outcomes?
\end{enumerate}

People who are new to keeping a research notebook will find it onerous. But overtime, they will find it useful and develop their own styles of writing research notebook that work for them.
\\

The writing in the research notebook can also be re-used for research papers.
\\

For writing Latex, I found mathpix to be extremely useful. 
\\

I am not sure what is the best tool to write a research notebook in. Latex is good for math, but bad for pictures. Other tools seem good for pictures but bad for math. If you know a tool that is good for both, let me know.

\end{document}
 