\documentclass[11pt]{article}

\usepackage[margin=0.3in]{geometry}
\usepackage{amsmath}
\usepackage{physics}
\usepackage{graphicx}

%Loading this so that mathbb is defined
\usepackage{amsfonts}

% This is required for triangleq
\usepackage{amssymb}

\begin{document}

Interpreting experimental result is one of the most important step in advancing a research agenda. Successful interpretations lead to discovery of bugs, provide insights into the research problems and motivations for additional follow-up experiments or additional concrete research questions that can be pursued. 
\\

The interpretation consists mostly of two steps:

\begin{enumerate}
	\item Deciding what statistics and curves to plot from the finished experiments.
	\item Connect the observed statistics and curves back into the research question that experiments were motivated by.
\end{enumerate}

In step 1, there are two issues we would like to tackle:

\begin{enumerate}
	\item Minimize the amount of time spent on deciding what statistics and curves to plot.
	\item Ensure we do not forget to check up on important statistics. When running complex experiments, we might be looking out for the behavior of many variables to understand the experiments. The more number of variables, the easier it is to forget to check up on one of them.
\end{enumerate}

An effective method is to create a checklist of statistics and curves to look at after each experiment. This ensures that both issues above are dealt with: after each experiment, we can just go down the checklist and look at the behavior of the variables listed in the checklist. This eliminates the mental overhead associated with issue 1 and the anxiety associated with issue 2.
\\

This works because for each research direction, after an initial period of formulation, the experiments to be run and the statistics that need to be studied remain roughly constant. Of course, we must modify the checklist accordingly when new variables needed to be added to the checklist as research progresses.

\end{document}
 